\documentclass{settings/notex}

% PREAMBLE ─────────────────────────────────────────────────────────────────────

% Images path
\graphicspath{{images/}}

% Additional packages here
\usepackage{blindtext}
\usepackage{siunitx}

% Subfiles package (best loaded last in the preamble)
\usepackage{subfiles}

% TITLE, AUTHOR AND DATE ───────────────────────────────────────────────────────

% Title, Author and Date
\title{
  Bag of Visual Words Classifier\\
  \vspace{0.35cm}
  \fontsize{12pt}{12pt}\selectfont{
    Computer Vision and Pattern Recognition Exam\\
    \vspace{0.25cm}
    University of Trieste (UniTS)
  }
}

\author{Marco Tallone}
\date{January 2025}

% DOCUMENT ─────────────────────────────────────────────────────────────────────

\begin{document}

\maketitle

\begin{abstract}
\noindent
This report presents the implementation of a Bag of Visual Words (BoW) image classifier.
The objective is to build a classifier for scene recognition by building a visual vocabulary from a set of images, representing them as normalized histograms of visual words, and performing multi-class classification.
The visual vocabuilary is built by clustering SIFT descriptors extracted from the images, and the classification is performed comparing K-Nearest Neighbors (KNN) and Support Vector Machines (SVM) classifiers.
Different kernels have been tested for the SVM classifier, including radial basis function (RBF), $\chi^2$ kernel and pyramid match kernel.
Results show that the best performance is achieved by the SVM classifiers, in particular when implemented with a spatial pyramid feature representation to add spatial information to the classic BoW approach.
\end{abstract}

% \tableofcontents

% Sections
\subfile{sections/01-introduction}
\subfile{sections/02-dataset-description}
\subfile{sections/03-visual-vocabulary-construction}
\subfile{sections/04-image-representation}
\subfile{sections/05-classification}
\subfile{sections/06-results}
\subfile{sections/07-conclusions}

% Bibliography
% Remember to compile with the sequence:
% lualatex -> bibtex -> lualatex -> lualatex
% \pagebreak
% \bibliographystyle{plainurl}
% \bibliography{bibliography}

% Appendix
\pagebreak
\appendix
\renewcommand{\thesection}{\Alph{section}} % Section numbering to A, B, C, ...
\renewcommand{\thesubsection}{\thesection\arabic{subsection}} % Subsection numbering to A1, A2, ...
\subfile{sections/appendix}

\end{document}
