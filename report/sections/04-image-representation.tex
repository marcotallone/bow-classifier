\documentclass[../main.tex]{subfiles}
\graphicspath{{\subfix{../images/}}} % Images path

\begin{document}

\section{Image Representation}\label{sec:image-representation}

A important step in the Bag of Visual Words (BoW) pipeline is the representation of images as normalized histograms of visual words.\\
In this phase, the SIFT descriptors previously extracted from the images are assigned to the visual words in the vocabulary by means of a clustering algorithm. Then, for each image in the dataset, a histogram is built by counting the occurrences of each visual word in the given image.\\
The result is a fixed-length representation in which images are seen as normalized histograms having $k$ bins, each corresponding to a visual word in the vocabulary. Such representation is used as input to the classifiers in section \ref{sec:classification} for the scene recognition task.\\
Additionally, in order to account for the relevance of the visual words, the \itt{term frequency-inverse document frequency (TF-IDF)} weighting scheme has also been implemented and compared. In this second case, the elements of the histograms representation have been computed as:

\begin{equation}
    t_i = \frac{n_{id}}{n_{d}} \cdot \log\left(\frac{N}{N_i}\right)
\end{equation}

where $n_{id}$ is the number of occurrences of the $i$-th visual word in image $d$, $n_{d}$ is the total number of visual words in image $d$, $N$ is the total number of images in the dataset, and $N_i$ is the number of images containing the $i$-th visual word.\\

\end{document}

