\documentclass[../main.tex]{subfiles}
\graphicspath{{\subfix{../images/}}} % Images path

\begin{document}

\section{Classification}\label{sec:classification}

In order to perform the scene recognition task, two main classes of classifiers have been implemented: K-Nearest Neighbors (KNN) and Support Vector Machines (SVM). 

\subsection{K-Nearest Neighbors (KNN)}\label{subsec:knn}

A K-Nearest Neighbors classifier taking as input features the normalized
histograms of visual words has been implemented. The classifier assigns to each
image in the test set the label of the majority class among its $k$ nearest
neighbors\footnote{Notice $k$ is the number of KNN neighbors, not to be confused with the number of clusters $K$.} in the training set.\\
In the simple case of $k=1$, the label of the closest histogram in the training set is assigned to the test image. 
A slightly better result can be achieved by performing a linear search for the
hyperparameter $k$ over the range $[1, 50]$ using the average accuracy as assessment metric. For each value of $k$ in the range, the accuracy of the corresponding KNN classifier is computed and stored. At the end, the value of $k$ that maximizes the accuracy is selected and the performance of the resulting classifier is evaluated. 

\subsection{Support Vector Machines (SVM)}\label{subsec:svm}

A series of multi-class Support Vector Machine (SVM) classifiers have been implemented
following the ``\textit{one-vs-all}'' strategy. For each possible class, a
single binary classifier is trained and the final prediction is obtained by
selecting the class with the highest confidence score. Each binary classifier is
trained taking as input features one of the presented image representations and
modified ground truth labels where the class of interest is labeled as $+1$ and all other classes are labeled as $-1$.

\pagebreak
Different kernels for these SVM classifiers have been tested and compared.
Initially, the default radial basis function (RBF) kernel has been adopted. Its
performance has then been compared with the generalized Gaussian kernel (with
$\gamma = \nicefrac{1}{2}$) based on
the $\chi^2$ distance and the histogram
intersection kernel in equation~\ref{eq:chi2-int-kernels}, both widely
adopted in histogram comparison tasks.
\begin{equation}\label{eq:chi2-int-kernels}
	k_{\chi^2}(\mathbf{x}, \mathbf{x}') = \exp\left(-\gamma \sum_i \frac{({x}_i
	- {x}'_i)^2}{{x}_i + {x}'_i}\right) 
	\quad \quad
	k_{\cap}(\mathbf{x}, \mathbf{x}') = \sum_i \min({x}_i, {x}'_i)
\end{equation}

For the \itt{SPM} feature representation, the SVM classifiers only used
the histogram intersection kernel as anticipated in the previous section and
pyramid levels up to $L=2$.

\end{document}

