\documentclass[../main.tex]{subfiles}
\graphicspath{{\subfix{../images/}}} % Images path

\begin{document}

\section{Conclusions}\label{sec:conclusions}

In conclusion, this study has shown the implementation of a Bag of Visual Words
(BoW) classifier for scene recognition from the construction of a visual
vocabulary by performing clustering on SIFT descriptors to the classification of
images using machine learning techniques such as K-Nearest Neighbors (KNN) and Support Vector Machines (SVM).\\
From the results, it emerges that the SVMs have clearly outperformed the KNN
classifiers in terms of accuracy, but the latter at least surpassed the dummy
classifier performance used as baseline. In particular, the usage of specialized
kernels for the SVM classifiers has also shown to be beneficial with respect to the default RBF kernel.\\
Concerning the worse results recorded for the soft assignment techniques, these
might be explained the different approach used in this study with respect to the
original paper. Specifically, since \itt{k-means} has been
used instead of the \itt{radius based} clustering implemented by the original
authors, it's possible that the approximation used to define the region around
each centroid might have led to worse results.
Moreover, \itt{Van Gemert et al.}~\cite{gemert} conducted $k$-fold
cross-validation on the training set to find the optimal value of $\sigma$
for the Gaussian kernel,
while in this study the value has been set to $\sigma = 200$ based on the
results of the paper due to computational and time constraints. Additionally, as
done in this study,
the authors also performed clustering on a random subset of the extracted
descriptors, but it's possible that the differences in the subset size and in
the
representativeness of the sampled descriptors might have led to different
results.\\
An improvement and further extension of this project could in fact be to repeat
the random sampling and clustering multiple times with different random seeds in order to collect significant statistics
about the performance of all the techniques and hence perform a more robust
comparison.\\
Ultimately, the best accuracy has been achieved by the SVM classifiers using the
spatial pyramid approach proposed by \itt{Lazebnik et al}.~\cite{lazebnik}.
Although the final accuracy of $\SI{75.54}{\percent}$ doesn't quite match the
one obtained by the authors of the original paper, in this case the results are
satisfactory since they are, by a good margin, better than the ones of the other
classifiers. This confirms the importance of adding spatial
information to the classic BoW approach. Also in this case, the slight
differences in the final results might be due to the different parameters used
in the clustering process.
Nevertheless, the results obtained are still satisfactory and confirm the validity of the BoW approach for the scene recognition task.

\end{document}


